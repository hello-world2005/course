\documentclass{beamer}
\usepackage[UTF8]{ctex}
\usepackage{graphicx}
\usepackage{cancel}

\begin{document}
  \maketitle
  \begin{frame}{什么是 ChatGPT?}
    这论文不学几年计算机谁看得懂啊。。。

    以下介绍以 OpenAI 官方介绍以及其他从业人员的介绍为基础进行概括(把自己看得懂的翻译一遍 \\doge)。
  \end{frame}
  \begin{frame}{什么是 ChatGPT?}
    ChatGPT 是以 GPT-3 为基础,类似于 InstructGPT 的对话模型。

    信息技术必修一告诉我们人工智能有三种实现:
    \begin{itemize}
      \item 符号主义
      \item 联结主义
      \item 行为主义
    \end{itemize}

    ChatGPT 是联结主义和行为主义的结合。
  \end{frame}
  \begin{frame}{什么是 ChatGPT?}
    联结主义意味着大量的训练数据,那 ChatGPT 是怎么训练的?
    \pause

    \begin{figure}[h]
      \includegraphics[scale=0.3]{ChatGPT_Diagram.jpg}
    \end{figure}
  \end{frame}
  \begin{frame}{为什么他是开创性的}
    让我们感觉它更像是一个真人?
  \end{frame}
  \begin{frame}{首先,他能干什么}
    \begin{itemize}
      \item 信息整合
      \item 帮你写代码
      \item 帮你写作文
      \item 自然语言聊天
      \item ……
    \end{itemize}
  \end{frame}
  \begin{frame}{讲个鬼故事}
    \includegraphics[scale=0.15]{qwq1.jpg}
  \end{frame}
  \begin{frame}{讲个鬼故事}
    \includegraphics[scale=0.12]{qwq2.jpg}
  \end{frame}
\end{document}
