\documentclass{beamer}
\usepackage[UTF8]{ctex}

\title{乔治·奥威尔}
\subtitle{坚定的民主社会主义者}
\date{\today}

\begin{document}
  \maketitle
  \begin{frame}{谁是奥威尔}
    英国左翼作家,新闻记者和社会评论家。

    自称为坚定的民主社会主义者。

    由于历史上东西方的对峙,乔治·奥威尔的作品经常被视为反苏和反共的代名词,因而曾在苏联、东欧、中国等社会主义国家遭到封杀。
  \end{frame}
  \begin{frame}{谁是奥威尔}
    《动物庄园》和《一九八四》为奥威尔的传世作品,他在书中以辛辣的笔触讽刺泯灭人性的极权主义社会和追逐权力者;而小说中对极权主义政权的预言在之后的五十年中也不断地与历史相印证,所以两部作品堪称世界文坛政治讽喻小说的经典之作。

    《一九八四》直到1979年才有简体中文版刊行,且出版初期被列为“内部读物”,只允许特定人群购买阅读,1985年允许大众阅读。
  \end{frame}
  \begin{frame}{生平}
    1903 年生于印度。
    \pause

    1917 年,奥威尔依靠自己的努力考取奖学金,进入伊顿公学。但他穷学生的背景使他备受歧视。早年的经历令他同情社会底层,呼唤平等和人性解放思想的形成和对极权主义的认识有着极其重要的影响。
    \pause

    1936 年参加西班牙内战。1937 年接纳了奥威尔的马克思主义统一工人党被斯大林控制的共产国际认定为托派组织,斯大林下令消灭马统工党,在共和军中建立恐怖统治。
    \pause

    1944 年写成《动物庄园》,标志着他的文字从单纯地关注底层社会的生活,转向了捍卫真正的民主社会主义。
    \pause

    1948 年写成《一九八四》。
  \end{frame}
  \begin{frame}{写作目的}
    \begin{quote}
      \kaishu 自1936年以来,我所写的每一行严肃作品,都是直接或间接反对极权主义,支持我所理解的民主社会主义。
    \end{quote}
  \end{frame}
  \begin{frame}{民主社会主义}
    这东西不好说,没有特别明确的概念。
    \pause

    “民主”通常意味着普选、多党制、司法独立、政治自由,反对法西斯主义和斯大林主义的一党专政。而社会主义经济则需要生产资料公有制,经济模式可以为计划经济、参与型经济或者市场社会主义。在实际执政过程中,许多民主社会主义者允许了多样型经济发展,并没有完全取缔市场经济,并着重于提供良好的福利保障和财富的再分配。
    \pause

    发展社会主义民主。反对斯大林式极权。
  \end{frame}
\end{document}
