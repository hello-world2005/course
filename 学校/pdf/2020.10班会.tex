\documentclass{beamer}
\usepackage[UTF8]{ctex}

\title{1984}
\subtitle{都怪 kq}
\date{\today}

\begin{document}
  \maketitle
  \begin{frame}
    这是一本反乌托邦的小说。
  \end{frame}
  \begin{frame}{背景}
    \begin{itemize}
      \item 上等阶级的核心党员,少数的统领精英,占全国人口的 2\% 左右;
      \item 中等阶级的外围党员,占全国人口的 13\% 左右;
      \item 下等阶级的无产阶级,占全国人口的 85\% 左右,代表没受过教育的无产阶级。
    \end{itemize}
  \end{frame}
  \begin{frame}{背景}
    \begin{itemize}
      \item 和平部 \ 战争
      \item 富裕部 \ 挨饿
      \item 友爱部 \ 拷打
      \item 真理部 \ 造谣
    \end{itemize}
  \end{frame}
  \begin{frame}{情节}
    概括题我最不擅长了。
    
    温斯顿·史密斯(小说的主人公)对英社思想上反叛、他与裘莉亚(可能算是女主?)偷情;及必然地在仁爱部被思想警察进行监禁、审问、拷打、洗脑和再教育。 
  \end{frame}
  \begin{frame}{什么是英社}
    小说中英国社会主义的新话写法。

    主要表现为权力的高度集中,箝制人民的言论、思想与创作自由,并借由重写语言、篡改历史来加强统治。

    想象一下一个不能够表达“自由”的语言。

    人们不能理解何谓自由,而且也找不到合适的用词来称呼它。
    
    除此之外,英国社会主义还要求人民接受“双重思想”概念,即同时接受两种相互违背的信念。在这样的概念下,无论党或老大哥做了什么,都将会被视为是正确,即使其行为相互矛盾亦然。 
  \end{frame}
  \begin{frame}{一本讽刺苏联的小册子}
    老大哥

    书中自始至终没有真正出现这号人物,也无法确信他是否真正存在,但大洋国的人民坚信他存在,他的存在也是权力的象征。 

    斯大林
  \end{frame}
  \begin{frame}{一本讽刺苏联的小册子}
    爱麦虞埃尔·果尔德施坦因

    书中虚构书籍《寡头政治体系的理论和实践》的作者。早年大洋国社会主义革命的领导者之一,但根据官方说法,他后来背叛革命成为国家的象征敌人,但实则可能因与老大哥的权力斗争而与老大哥反面。

    托洛茨基
  \end{frame}
  \begin{frame}{一本讽刺苏联的小册子}
    奥勃良

    思想警察头目,捕捉那些思想不纯洁的人。

    温斯顿·史密斯被捕后,奥勃良负责在友爱部对他进行审问。

    格别乌
  \end{frame}
  \begin{frame}{一本讽刺苏联的小册子}
    非人

    “蒸发”的人,被杀,且在历史上或记忆上抹去这个人的存在。

    大清洗以及暗杀。
  \end{frame}
  \begin{frame}{不仅仅是一本讽刺苏联的小册子}
    民族主义

    正面、负面、转向

    为政治服务的内宣
  \end{frame}
  \begin{frame}{不仅仅是一本讽刺苏联的小册子}
    审查
    
    消除非人

    仅限官方出版物

    修改历史报道
  \end{frame}
  \begin{frame}{不仅仅是一本讽刺苏联的小册子}
    监控
  \end{frame}
  \begin{frame}{不仅仅是一本讽刺苏联的小册子}
    自由意志

    绝对忠诚
  \end{frame}
  \begin{frame}{不仅仅是一本讽刺苏联的小册子}
    \begin{quotation}
      \noindent
      我最近的小说《一九八四》不是为了攻击社会主义或我支持的工党,而是揭露扭曲……部分已经可以在共产主义和法西斯主义中领会到……这本书的场景放在英国,是为了强调英语民族并非天生比其他民族优秀,并且,如果不与极权主义做斗争,它将无往不胜。
    \end{quotation}
    奥威尔描述自己是一个民主社会主义者。
  \end{frame}
  \begin{frame}
    “我怎么能不看到眼前的东西呢?二加二等于四呀。” “有时候是四,温斯顿。但有时候是五。有时候是三。有时候全是。你得再努力一些。要神志健全,不是容易的事。” 

    有真理,就有非真理,如果你坚持真理;哪怕全世界都不同意你,你也没有发疯。 

    古代的各种文明都话自己是建筑在博爱和正义之上的。我们的文明则建筑在仇恨上。在我们的世界里,除了恐惧、狂怒、得意、自贬以外,没有别的感情。其他一切都要摧毁。 

    不再有好奇心,不再有生命过程的应用。一切其他乐趣都要消灭掉。但是,温斯顿,请你不要忘了,对于权力的沉醉,却永远存在,而且不断地增长,不断地越来越细腻。每时每刻,永远有胜利的欢悦,践踏束手待毙的敌人的快澸。如果你要设想一幅未来的图景,就想象一只脚踩在一张人脸上好了——永远如此。
  \end{frame}
  \begin{frame}
    可以不怀好意地对应一下(雾
  \end{frame}
\end{document}
